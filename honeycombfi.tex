\documentclass[%
 reprint,
 amsmath,amssymb,
 aps,
 pra, %uses citation style more similar to current PRB
%prb,
%rmp,
%prstab,
%prstper,
%floatfix,
]{revtex4-1}

\usepackage{graphicx}
\usepackage{dcolumn}
\usepackage{bm}
\usepackage{color}
\usepackage{hyperref}

\newcommand{\dd}{\mathrm{d}}
\newcommand{\DD}{\mathrm{D}}
\newcommand{\ee}{\mathrm{e}}
\newcommand{\ii}{\mathrm{i}\,}
\newcommand{\TT}{\mathcal{T}}

\newcommand{\sgn}{\mathrm{sgn}}
\newcommand{\Tr}{\mathrm{Tr}}
\newcommand{\dg}{\dagger}

\newcommand{\mc}{\mathcal}


\begin{document}

\title{Schwinger boson representation for featureless insulators on spin-1/2 honeycomb lattice}


%\author{Shenghan Jiang}
%\email{jiangsh@caltech.edu}
%\affiliation{Department of Physics and Institute for Quantum Information and Matter, \\
%California Institute of Technology, Pasadena, California 91125, USA}

\date{\today}

\begin{abstract}
  
\end{abstract}

\maketitle

\tableofcontents

\section{Introduction}


\section{Schwinger boson construction}
In this section, we argue that conventional Schwinger boson formalism fails to give mean field ansatz for featureless insulators on spin-1/2 honeycomb lattice.
Instead, we provide a novel two-orbital Schwinger boson formalism, which helps to construct ansatz for featureless insulators.

\subsection{Single orbital Schwinger boson and monopole issues}

\subsection{Two-orbital Schwinger boson formulation for spin-1/2}
We introduce two orbitals of Schwinger bosons $b_{\alpha s}$, where $\alpha=1,2$ labels orbital, $s=\uparrow,\downarrow$ labels spin. 
The spin operator is represented by these Schwinger bosons as
\begin{align}
  \vec{S}=\sum_{\alpha=1}^2\vec{S}_\alpha=\sum_{\alpha=1}^2\frac{1}{2}\sum_{s,s'=\uparrow,\downarrow}b_{\alpha s}^\dg\vec{\sigma}_{ss'}b_{\alpha s'}
  \label{}
\end{align}
The Schwinger boson representation leads to an enlarged Hilbert space, and one should project out unphysical states to represent physical spin-1/2 states.
There are several constraints for physical states. 
The first constraint is chosen to be that the total boson number is fixed to be three per site:
\begin{align}
  n=\sum_{\alpha}n_\alpha=\sum_{\alpha s}b_{\alpha s}^\dg b_{\alpha s}=3
  \label{}
\end{align}
There are totally 20 states satisfying this condition.
It is easy to check that these 20 states can be organized to four spin-3/2's and two spin-1/2's: 
\begin{align}
  \left( \bigoplus_{i=1}^4\frac{3}{2} \right) \oplus \left(  \bigoplus_{i=1}^2\frac{1}{2} \right)
  \label{}
\end{align}
The highest weight states for spin-3/2's are
\begin{align}
  &\frac{1}{\sqrt{6}}\left(b_{1\uparrow}^\dg \right)^3|0\rangle~,\quad\frac{1}{\sqrt{6}}\left(b_{2\uparrow}^\dg \right)^3|0\rangle~,\notag\\
  &\frac{1}{\sqrt{2}}\left( b_{1\uparrow}^\dg \right)^2 b_{2\uparrow}^\dg|0\rangle~,\quad \frac{1}{\sqrt{2}}b_{1\uparrow}^\dg \left(b_{2\uparrow}^\dg\right)^2|0\rangle~. 
  \label{}
\end{align}
And the highest weight states for spin-1/2's are
\begin{align}
  \frac{1}{\sqrt{3}}\,b_{1\uparrow}^\dg\left( b_{1\uparrow}^\dg b_{2\downarrow}^\dg-b_{1\downarrow}^\dg b_{2\uparrow}^\dg \right)|0\rangle\notag\\
  \frac{1}{\sqrt{3}}\,b_{2\uparrow}^\dg\left( b_{1\uparrow}^\dg b_{2\downarrow}^\dg-b_{1\downarrow}^\dg b_{2\uparrow}^\dg \right)|0\rangle
  \label{}
\end{align}

We define the orbital pseudo-spins $\vec{T}$ as
\begin{align}
  \vec{T}=\sum_{s,\alpha,\beta}\frac{1}{2}b_{\alpha s}^\dg\vec{\tau}_{\alpha\beta}b_{\beta s}
  \label{}
\end{align}
where $\vec{\tau}$ are orbital Pauli matrix. 
Thus, these 20 states can be further organized to $SU(2)_{spin}\times SU(2)_{obit}$ representation $\left( \frac{3}{2},\frac{3}{2} \right)\oplus\left( \frac{1}{2},\frac{1}{2} \right)$.
The physical spin-1/2 states are contained in sector $\left( \frac{1}{2},\frac{1}{2} \right)$. 

In $(S^z,T^z)$ eigenbasis, these four states read
\begin{align}
  \begin{pmatrix}
  \frac{1}{\sqrt{3}}\,b_{1\uparrow}^\dg\left( b_{1\uparrow}^\dg b_{2\downarrow}^\dg-b_{1\downarrow}^\dg b_{2\uparrow}^\dg \right) & 
  \frac{1}{\sqrt{3}}\,b_{2\uparrow}^\dg\left( b_{1\uparrow}^\dg b_{2\downarrow}^\dg-b_{1\downarrow}^\dg b_{2\uparrow}^\dg \right) \\[0.4cm]
  \frac{1}{\sqrt{3}}\,b_{1\downarrow}^\dg\left( b_{1\uparrow}^\dg b_{2\downarrow}^\dg-b_{1\downarrow}^\dg b_{2\uparrow}^\dg \right) & 
  \frac{1}{\sqrt{3}}\,b_{2\downarrow}^\dg\left( b_{1\uparrow}^\dg b_{2\downarrow}^\dg-b_{1\downarrow}^\dg b_{2\uparrow}^\dg \right) 
  \end{pmatrix}
  |0\rangle
  \label{eq:spin-1/2_orbit-1/2_sector}
\end{align}
where rows corresponds to spin $\pm1/2$, and columns corresponds to orbital $\pm1/2$.
The physical spin-1/2 states can be chosen as an arbitrary linear combination of the columns in Eq.~(\ref{eq:spin-1/2_orbit-1/2_sector}):
\begin{align}
  |s_p\rangle&=\ee^{\ii\varphi}\left(\cos\left( \frac{\theta}{2} \right)|s_1\rangle+\ee^{\ii\phi}\sin\left( \frac{\theta}{2} \right)|s_2\rangle \right)\notag\\
  &\equiv (|s_1\rangle, |s_2\rangle) \cdot \begin{pmatrix}w_1\\w_2\end{pmatrix} \label{}
\end{align}
where $w_1=\ee^{\ii\varphi}\cos(\theta/2)$, $w_2=\ee^{\ii\varphi}\ee^{\ii\phi}\sin(\theta/2)$ with $\theta\in[0,\pi]$, $\phi,\varphi\in[0,2\pi)$.
$|s_i\rangle$ denotes spin state constructed by operators of $i$th column in Eq.~(\ref{eq:spin-1/2_orbit-1/2_sector}). 

For this choice, we get additional constraint as
\begin{align}
  \vec{n}\cdot\vec{T}=\frac{1}{2}
  \label{}
\end{align}
where $\vec{n}=\left( \sin\theta\cos\phi,\sin\theta\sin\phi,\cos\theta \right)$.

For a given $\vec{n}$, the gauge degree of freedom is $U(1)\times U(1)$:
\begin{align}
  b\rightarrow \ee^{-\ii\theta}\,b~,\quad b\rightarrow\ee^{-\ii\phi(\vec{n}\cdot\vec{\tau})}\,b
  \label{eq:parton_gauge_freedom}
\end{align}
To see this, we re-express physical states as
\begin{align}
  \begin{pmatrix} |\uparrow_p\rangle\\[0.2cm]|\downarrow_p\rangle\end{pmatrix}=
  (b^\dg\cdot w_{\vec{n}})(b^\dg\cdot \sigma^y\tau^y \cdot b^*) |0\rangle
  \label{eq:parton_phys_state}
\end{align}
where $w_{\vec{n}}=\sigma^0\otimes\begin{pmatrix}w_1\\w_2\end{pmatrix}$, which acts nontrivially on orbital space. 
Notice that $w_{\vec{n}}$ is invariant (up to a phase factor) under operation rotation along $\vec{n}$ axis: $\ee^{\ii\phi}w_{\vec{n}}=\ee^{\ii\phi(\vec{n}\cdot\vec{\tau})}\cdot w_{\vec{n}}$.
The second term in Eq.~(\ref{eq:parton_phys_state}) is a spin and orbital singlet, which is invariant under orbital $SU(2)$ transformation.
We then conclude that gauge transformations defined in Eq.~(\ref{eq:parton_gauge_freedom}) leaves physical states invariant up to overall phase factors. 
These overall phase factors can be interpreted as the background gauge charges carried by physical states.
Under gauge transformation in Eq.~(\ref{eq:parton_gauge_freedom}), we have
\begin{align}
  |s_p\rangle\rightarrow\ee^{\ii 3\theta}|s_p\rangle~,\quad |s_p\rangle\rightarrow\ee^{\ii \phi}|s_p\rangle
  \label{}
\end{align}
respectively.
In other words, physical states carry background charge $(3,1)$ under $U(1)\times U(1)$ gauge group.

In the following, we focus on the case where $w=(1,0)$. 
In this case, gauge transformation reads
\begin{align}
  b_{\boldsymbol{r}}\rightarrow \ee^{\ii\theta(\boldsymbol{r})}\,b_{\boldsymbol{r}},\quad b_{\boldsymbol{r}}\rightarrow\ee^{\ii\phi(\boldsymbol{r})\tau^z}\,b_{\boldsymbol{r}}
  \label{}
\end{align}
where $\boldsymbol{r}$ labels the lattice site.
Notice that gauge transformation is in general site-dependent.


\section{Classification of featureless insulators by $PSG$}

\section{Neighbouring magnetic ordered phases and possible phase transitions}

\section{Conclusions}


\appendix

\section{Details of $PSG$ analysis}
\subsection{Global symmetry group}
\subsection{$PSG$ equations}
\subsection{Constraint on mean field ansatz}

\section{Monopole quantum number calculation}



\bibliography{bibhoneycombfi} % Produces the bibliography via BibTeX.

\end{document}

